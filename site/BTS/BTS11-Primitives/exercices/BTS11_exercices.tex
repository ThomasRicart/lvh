\documentclass[a4paper,10pt]{report}

\def\packagepath{../../../../../preambule}   % path du package principal
\usepackage{\packagepath/preambule}    % utilisation du fichier de configuration

\def\level{BTS }              % Classe
\def\course{CIEL1}           % Matière
\def\eval{BTS11 -- Les Primitives}

\def\visibleornot{visible}    % visible or invisible
\def\documentpath{./Sources_Latex}
\def\date{}
\renewcommand{\arraystretch}{1.2}

\def\ptsexoA{5}
\def\ptsexoB{5}
\def\ptsexoC{5}
\def\ptsexoD{5}

\def\ptstotal{\ptsexoA+\ptsexoB+\ptsexoC+\\ptsexoD}

\usepackage{hyperref}
\usepackage{float}
\usepackage[T1]{fontenc}

\begin{document}

\pagestyle{DS_LP}
\NomPrenom{}

\bigskip

%%=============================================================
\exods{0} \medskip

\textbf{Calculs de Primitives Usuelles}: 
Déterminer une primitive $F$ pour chacune des fonctions suivantes définies sur $\mathbb{R}$. N'oubliez pas la constante d'intégration $k$.

\begin{enumerate}
    \item $f(x) = 4x^3 - 6x^2 + 2$ 
    \item $g(x) = 5e^x$
    \item $h(x) = 3\cos(x) + 2\sin(x)$
    \item $k(x) = e^{-2x}$
\end{enumerate}

\dotfill{} 
\bigskip

%%=============================================================
\exods{0} \medskip

\textbf{Primitive particulière et Exponentielle}: 
On considère la fonction $f$ définie sur $\mathbb{R}$ par : 
\begin{center}
    $f(t) = 12e^{-4t}$
\end{center}

\begin{enumerate}
    \item Déterminer la forme générale des primitives $F$ de la fonction $f$ sur $\mathbb{R}$.
    \dotfill{}
    \item Déterminer la primitive particulière $F_0$ qui vérifie la condition initiale : $F_0(0) = 10$.
    \dotfill{}
    \item En déduire l'expression complète de $F_0(t)$.
    \dotfill{}
\end{enumerate}

\bigskip

%%=============================================================
\exods{0} \medskip

\textbf{Condition Initiale (Logarithme)}: 
Soit la fonction $g$ définie sur $]0 ; +\infty[$ par $g(x) = \dfrac{1}{x} + 1$.

\begin{enumerate}
    \item Donnez la forme générale des primitives $G$ de la fonction $g$.
    \dotfill{}
    \item Déterminez la primitive particulière $G_0$ telle que $G_0(1) = 5$.
    \dotfill{} 
    \item Calculez la valeur exacte de $G_0(e)$.
    \dotfill{}
\end{enumerate}

\bigskip

%%=============================================================
\exods{0} \medskip

\textbf{Application aux Sciences Physiques}: 
Dans un circuit électronique (charge d'un condensateur), l'intensité du courant est donnée par $i(t) = C \dfrac{du}{dt}$. On considère un composant où l'intensité est $i(t) = 0,05 \sin(100\pi t)$ avec $C = 500 \mu F$.

\begin{enumerate}
    \item Sachant que la tension $u(t)$ est une primitive de $\dfrac{i(t)}{C}$, déterminez l'expression de $u(t)$.
    \dotfill{}
    \item Si le condensateur est initialement déchargé ($u(0) = 0$), déterminez la valeur de la constante d'intégration.
    \dotfill{}
\end{enumerate}

\bigskip

\end{document}