\documentclass[a4paper,10pt]{report}

\def\packagepath{../../../../../preambule}   % path du package principal
\usepackage{\packagepath/preambule}    % utilisation du fichier de configuration

\def\level{BTS }              % Classe
\def\course{CIEL1}           % Matière
\def\eval{BTS10 -- Les Nombres Complexes}

\def\visibleornot{visible}    % visible or invisible
\def\documentpath{./Sources_Latex}
\def\date{}
\renewcommand{\arraystretch}{1.2}

\def\ptsexoA{5}
\def\ptsexoB{5}
\def\ptsexoC{5}
\def\ptsexoD{5}

\def\ptstotal{\ptsexoA+\ptsexoB+\ptsexoC+\ptsexoD}

\usepackage{hyperref}
\usepackage{float}
\usepackage[T1]{fontenc}

\begin{document}

\pagestyle{DS_LP}
\NomPrenom{}

\bigskip

%%=============================================================
\exods{0} \medskip

\textbf{Compréhension et Forme Algébrique}: Soit un nombre complexe $z$ défini par :
\begin{center}
    $z = 3 - 4i$
\end{center}

\begin{enumerate}
    \item \textbf{Terminologie} : Identifiez la partie réelle $Re(z)$ et la partie imaginaire $Im(z)$ de ce nombre. [cite: 283]
    \dotfill{}
    \item \textbf{Conjugué} : Donnez l'écriture du nombre conjugué $\overline{z}$. 
    \dotfill{}
    \item \textbf{Module} : Calculez le module $|z|$. 
    \dotfill{}
    \item \textbf{Opposé} : Donnez l'affixe du point $P$ symétrique du point $M(z)$ par rapport à l'origine. 
    \dotfill{}
\end{enumerate}

\bigskip

%%=============================================================
\exods{0} \medskip

\textbf{Opérations et Puissances de $i$}: 
Complétez le script ou les égalités suivantes en simplifiant au maximum pour obtenir une forme $a+ib$. 

\begin{enumerate}
    \item $i^2 = \dots\dots$
    \item $i^3 = \dots\dots$
    \item $(1+i)^2 =  \dots\dots$
    \item $z = \dfrac{3-i}{4-3i}$. 
\end{enumerate}

\bigskip

%%=============================================================
\exods{0} \medskip

\textbf{Résolution d'Équations}: 
On souhaite résoudre dans $\mathbb{C}$ l'équation du second degré suivante : [cite: 335]
\begin{center}
    $4z^{2}-4z+5=0$
\end{center}

\begin{enumerate}
    \item Calculez le discriminant $\Delta = b^2 - 4ac$ :
    \dotfill{} 
    \item En déduire la nature des solutions (réelles ou complexes conjuguées) :
    \dotfill{} 
    \item Donnez les valeurs exactes des solutions $z_1$ et $z_2$ :
    \dotfill{} 
\end{enumerate}

\bigskip
%%=============================================================
\exods{0} \medskip

\textbf{Formes Trigonométrique et Exponentielle}: 
Soit le nombre complexe $z_A = -2 + 2i$. 

\begin{enumerate}
    \item Calculez le module $r = |z_A|$ :
    \dotfill{} 
    \item Déterminez un argument $\theta$ en utilisant les formules $\cos \theta = \dfrac{a}{r}$ et $\sin \theta = \dfrac{b}{r}$ :
    \dotfill{} 
    \item Écrivez $z_A$ sous forme exponentielle ($r e^{i\theta}$) :
    \dotfill{} 
    \item \textbf{Application Électronique} : Si une impédance complexe est $Z = 10 + 10i$, donnez sa forme exponentielle :
    \dotfill{}
\end{enumerate}

\end{document}