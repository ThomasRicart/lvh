\documentclass[a4paper,10pt]{report}

\def\packagepath{../../../../../preambule}   % path du package principal
\usepackage{\packagepath/preambule}    % utilisation du fichier de configuration

\def\level{NSI }              % Classe
\def\course{Première}           % Matière
\def\eval{WEB06 -- Web -- IHM Capteurs / Actionneurs}

\def\visibleornot{visible}    % visible or invisible
\def\documentpath{./Sources_Latex}
\def\date{}
\renewcommand{\arraystretch}{1}  % Ecart dans les tableaux

\def\ptsexoA{0}
\def\ptsexoB{0}

\def\ptstotal{\ptsexoA+\ptsexoB}

\usepackage{hyperref}

\usetikzlibrary{shapes, arrows, positioning}
\usetikzlibrary{shapes.geometric, arrows, positioning, decorations.pathreplacing}

\tikzstyle{etat} = [draw, rounded corners=15pt, minimum width=2.5cm, minimum height=1.2cm, text centered, font=\bfseries]
\tikzstyle{fleche} = [thick, ->, >=stealth]

\usepackage{float}
\usepackage[T1]{fontenc}

\lstset{
    language=HTML,
    basicstyle=\ttfamily\small,
    breaklines=true,
    frame=single,
    backgroundcolor=\color{gray!5},
    keywordstyle=\color{blue},
    tagstyle=\color{blue},
    commentstyle=\color{green!50!black},
    stringstyle=\color{red},
    tabsize=2,
    showstringspaces=false
}

\begin{document}

\renewcommand{\labelitemi}{\textbullet}

\pagestyle{DS_LP}
\NomPrenom{}

%%=============================================================
\exods{0}

\medskip
\medskip

\textbf{La boucle de contrôle d'un système domotique}

Un système connecté repose sur trois composants clés : le \textbf{Capteur}, le \textbf{Traitement} et l'\textbf{Actionneur}.



\begin{enumerate}
    \item Reliez chaque composant à sa définition :
    \begin{itemize}
        \item \textbf{Capteur} $\bullet$ \hfill $\bullet$ Agit sur le monde réel (ex: allumer un ventilo).
        \item \textbf{Traitement} $\bullet$ \hfill $\bullet$ Mesure une grandeur physique (ex: température).
        \item \textbf{Actionneur} $\bullet$ \hfill $\bullet$ Analyse les données et prend une décision (JavaScript).
    \end{itemize}
    
    \item Dans notre projet Dashboard, le curseur (\texttt{input type="range"}) simule quel composant ? \dotfill{}
\end{enumerate}

\bigskip

\exods{0}

\textbf{Analyse des flux de données}

Complétez le tableau suivant pour identifier la nature des échanges dans votre interface :

\begin{center}
\renewcommand{\arraystretch}{1.5}
\begin{tabular}{|l|c|l|}
\hline
\textbf{Composant} & \textbf{Flux (Entrant / Sortant)} & \textbf{Rôle dans l'IHM} \\ \hline
Capteur de température & ........................ & Fournir la donnée brute. \\ \hline
Affichage LCD / Texte & ........................ & Informer l'utilisateur. \\ \hline
Moteur de ventilation & ........................ & Exécuter l'ordre de refroidissement. \\ \hline
\end{tabular}
\end{center}

\bigskip

\exods{0}

\textbf{Pilotage automatique d'un actionneur}

On souhaite automatiser la ventilation. Si la température (\texttt{temp}) dépasse 26°C, l'actionneur doit s'activer. Complétez le code JavaScript suivant :

\begin{lstlisting}[language=Java]
const affichageVentilo = document.getElementById("etat-ventilo");

if (temp >= 26) {
    affichageVentilo.textContent = ".........."; // Message d'activation
    affichageVentilo.style.color = ".........."; // Couleur d'alerte
} else {
    affichageVentilo.textContent = "ARRET";
    affichageVentilo.style.color = "grey";
}
\end{lstlisting}

\bigskip

\exods{0}

\textbf{Défi : L'IHM Bidirectionnelle}

L'objectif est de créer un bouton \textbf{"AUTO-CONFORT"}. Lorsqu'on clique dessus, le système doit forcer la température à 21°C.

\begin{enumerate}
    \item Quel attribut HTML devez-vous ajouter au bouton pour appeler une fonction au clic ? \dotfill{}    
    \item Complétez la fonction JavaScript correspondante :
\end{enumerate}

\begin{lstlisting}[language=Java]
function modeAutoConfort() {
    // 1. On cible le curseur (input range) par son ID
    const curseur = document.getElementById("temp-slider");

    // 2. On regle la valeur a 21
    curseur.value = ..........;

    // 3. On appelle la fonction de mise a jour du dashboard
    mettreAJourDashboard();
}
\end{lstlisting}
%%%%%%%%%%%%%%%%%%


\end{document}