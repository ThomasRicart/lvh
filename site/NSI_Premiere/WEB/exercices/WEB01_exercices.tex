\documentclass[a4paper,10pt]{report}

\def\packagepath{../../../../../preambule}   % path du package principal
\usepackage{\packagepath/preambule}    % utilisation du fichier de configuration

\def\level{NSI }              % Classe
\def\course{Première}           % Matière
\def\eval{WEB01 -- Web -- HTML }

\def\visibleornot{visible}    % visible or invisible
\def\documentpath{./Sources_Latex}
\def\date{}
\renewcommand{\arraystretch}{1}  % Ecart dans les tableaux

\def\ptsexoA{0}
\def\ptsexoB{0}

\def\ptstotal{\ptsexoA+\ptsexoB}

\usepackage{hyperref}

\usetikzlibrary{shapes, arrows, positioning}
\usetikzlibrary{shapes.geometric, arrows, positioning, decorations.pathreplacing}

\tikzstyle{etat} = [draw, rounded corners=15pt, minimum width=2.5cm, minimum height=1.2cm, text centered, font=\bfseries]
\tikzstyle{fleche} = [thick, ->, >=stealth]

\usepackage{float}
\usepackage[T1]{fontenc}

\lstset{
    language=HTML,
    basicstyle=\ttfamily\small,
    breaklines=true,
    frame=single,
    backgroundcolor=\color{gray!5},
    keywordstyle=\color{blue},
    tagstyle=\color{blue},
    commentstyle=\color{green!50!black},
    stringstyle=\color{red},
    tabsize=2,
    showstringspaces=false
}

\begin{document}

\renewcommand{\labelitemi}{\textbullet}

\pagestyle{DS_FP}
\NomPrenom{}

%%=============================================================
\exods{0}

\medskip


\textbf{Analyse de code et vocabulaire}

Observez l'extrait de code suivant et répondez aux questions:

\begin{lstlisting}
<section id="meteo">
    <h3>Previsions locales</h3>
    <p>Le ciel est <a href="https://meteofrance.com">nuageux</a>.</p>
    <img src="nuage.png" alt="Icone de nuage">
</section>
\end{lstlisting}


\begin{enumerate}
    \item \textbf{Identification} : Relevez une balise double et une balise orpheline dans ce code.
    \begin{itemize}
        \item Balise double : ............................................................
        \item Balise orpheline : .........................................................
    \end{itemize}
    
    \item \textbf{Attributs} : Quel est le nom et la valeur de l'attribut de la balise \texttt{<a>} ?
    \begin{itemize}
        \item Nom : ............................ | Valeur : ............................
    \end{itemize}
    
    \item \textbf{Accessibilité} : À quoi sert l'attribut \texttt{alt} présent sur la balise \texttt{<img>} ?
    \\[0.3cm] ..................................................................................................................................................
\end{enumerate}


\bigskip

\exods{0}

\textbf{Représentation du DOM (Arbre)}

Le \textbf{HTML} structure les données de manière hiérarchique sous forme d'arbre appelé \textbf{DOM}.

Dessinez l'arbre \textbf{DOM} correspondant au code de l'Exercice 1. Chaque balise est un noeud, les balises imbriquées sont des enfants.

\vspace{4cm}


\exods{0}

\textbf{Passage de l'Arbre au Code}

Écrivez le code \textbf{HTML} correspondant à l'arborescence suivante en respectant l'indentation.


\dirtree{%
    .1 body.
    .2 header.
    .3 h1 (texte: "Mon Dashboard").
    .3 nav (Barre de navigation).
    .4 a (lien vers "Accueil").
    .2 main.
    .3 div (Classe "card").
    .4 p (Texte: "Cuisine").
    .4 p (Texte: "Température:").
    .3 div (Classe "card").
    .4 h2 (Texte: "Garage").
    .4 ul (Liste d'états).
    .5 li (Texte: "Porte: Fermée").
    .5 li (Texte: "Lumière: Off").
    }


    \pagebreak
\thispagestyle{DS_LP}
    \exods{0}

    Voici le fichier \texttt{index.html} de départ. Il contient la structure minimale obligatoire.

    \begin{lstlisting}
<!DOCTYPE html>
<html lang="fr">
    <head>
        <meta charset="UTF-8">
        <title>Mon Dashboard</title>
    </head>
    <body>
        <header>
            <h1>Gestionnaire de Maison</h1>
        </header>

        <main>

        </main>
    </body>
</html>
\end{lstlisting}

En vous aidant de votre cours sur les balises, complétez le code ci-dessus pour remplir les missions suivantes :

\subsection*{Mission A : Structurer le Salon}
Dans la balise \texttt{<main>}, ajoutez une section pour le \textbf{Salon} contenant :
\begin{itemize}
    \item Un titre de niveau 2 : "Ambiance du Salon".
    \item Une liste à puces (\texttt{<ul>}) indiquant deux mesures : "Luminosité : 300 lux" et "Température : 19°C".
    \item Une image du salon (\texttt{salon.jpg}) avec un texte alternatif approprié.
\end{itemize}

\subsection*{Mission B : Ajouter des contrôles au Garage}
Créez une deuxième section sous la première pour le \textbf{Garage} :
\begin{itemize}
    \item Un titre de niveau 2 : "Sécurité Garage".
    \item Un bouton (\texttt{<button>}) affichant le texte "Ouvrir le portail".
    \item Un champ de saisie (\texttt{<input type="password">}) pour entrer un code secret.
\end{itemize}

\subsection*{Mission C : Navigation et Liens}
\begin{itemize}
    \item Dans le \texttt{<header>}, ajoutez un lien (\texttt{<a>}) vers le site \texttt{meteofrance.com} intitulé "Voir la météo".
    \item Ajoutez un pied de page (\texttt{<footer>}) tout en bas du document avec votre nom et la mention "Interface Domotique 2026".
\end{itemize}

\bigskip

\exods{0}

Un collègue a écrit le code suivant pour la section "Cuisine", mais il contient plusieurs erreurs de syntaxe et ne respecte pas la structure en arbre (DOM).

\begin{lstlisting}
<section>
    <h2>Controle de la Cuisine
    <p>Le four est actuellement : <strong>allume<p></strong>
    <ul>
        <li>Lumiere : On</li>
        <li>Cafetiere : Off
    </ul>
    <img src="cuisine.jpg"> 
</section>
\end{lstlisting}

\begin{enumerate}
    \item \textbf{Analyse} : Listez les erreurs que vous avez repérées.
    
    \item \textbf{Correction} : Réécrivez  le code correct.
\end{enumerate}



\end{document}