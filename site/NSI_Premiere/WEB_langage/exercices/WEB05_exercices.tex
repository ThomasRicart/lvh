\documentclass[a4paper,10pt]{report}

\def\packagepath{../../../../../preambule}   % path du package principal
\usepackage{\packagepath/preambule}    % utilisation du fichier de configuration

\def\level{NSI }              % Classe
\def\course{Première}           % Matière
\def\eval{WEB05 -- Web -- Java Script 02 }

\def\visibleornot{visible}    % visible or invisible
\def\documentpath{./Sources_Latex}
\def\date{}
\renewcommand{\arraystretch}{1}  % Ecart dans les tableaux

\def\ptsexoA{0}
\def\ptsexoB{0}

\def\ptstotal{\ptsexoA+\ptsexoB}

\usepackage{hyperref}

\usetikzlibrary{shapes, arrows, positioning}
\usetikzlibrary{shapes.geometric, arrows, positioning, decorations.pathreplacing}

\tikzstyle{etat} = [draw, rounded corners=15pt, minimum width=2.5cm, minimum height=1.2cm, text centered, font=\bfseries]
\tikzstyle{fleche} = [thick, ->, >=stealth]

\usepackage{float}
\usepackage[T1]{fontenc}

\lstset{
    language=HTML,
    basicstyle=\ttfamily\small,
    breaklines=true,
    frame=single,
    backgroundcolor=\color{gray!5},
    keywordstyle=\color{blue},
    tagstyle=\color{blue},
    commentstyle=\color{green!50!black},
    stringstyle=\color{red},
    tabsize=2,
    showstringspaces=false
}

\begin{document}

\renewcommand{\labelitemi}{\textbullet}

\pagestyle{DS_LP}
\NomPrenom{}

%%=============================================================
\exods{0}

\medskip
\textbf{Prendre des décisions en JavaScript}

Observez le bloc de code suivant qui analyse une mesure de température :

\begin{lstlisting}[language=Java]
let temperature = 26;

if (temperature >= 26) {
    console.log("Alerte Chaleur !");
} else {
    console.log("Temperature normale.");
}
\end{lstlisting}

\begin{enumerate}
    \item Quel message s'affichera dans la console si la température est de \textbf{22} ? \dotfill{}    
    \item Quel est le rôle du bloc \texttt{else} ? \dotfill{}
    
    \item Citez deux \textbf{opérateurs de comparaison} vus en cours: \dotfill{}
\end{enumerate}

\bigskip

\exods{0}

\textbf{Manipulation des classes CSS (\texttt{classList})}

Pour modifier l'apparence d'un élément, il est préférable d'utiliser les classes CSS plutôt que de modifier les styles un par un.

\begin{enumerate}
    \item Quelle instruction permet d'\textbf{ajouter} la classe "chaud" à l'élément \texttt{zoneCard} ?
    \\[0.2cm] \texttt{zoneCard.classList.........................("chaud");}

    \item Pourquoi est-il important d'utiliser \texttt{classList.remove("froid")} avant d'ajouter la classe "chaud" ?
    \\[0.2cm] ....................................................................................................
\end{enumerate}

\bigskip

\exods{0}

\textbf{Logique de seuils et Feedback visuel}

On souhaite modifier l'icône de statut du dashboard selon la température \texttt{t}. Complétez la structure conditionnelle suivante :

\begin{lstlisting}[language=Java]
if (t >= 26) {
    zoneCard.classList.add("chaud");
    icone.textContent = ".........."; // Icone feu
} else if (t <= 17) {
    zoneCard.classList.add(".........."); // Classe pour le froid
    icone.textContent = ".........."; // Icone neige
} else {
    icone.textContent = ".........."; // Icone OK
}
\end{lstlisting}

\bigskip

\exods{0}

\textbf{Application : Gestion de la lumière}

L'utilisateur clique sur un bouton pour éteindre les feux. Le code HTML est le suivant : 
\texttt{<button onclick="gererLumiere('eteindre')">Eteindre</button>}

Complétez la fonction JavaScript pour simuler l'extinction en ajoutant la classe \texttt{"mode-nuit"} au corps de la page (\texttt{body}) :

\begin{lstlisting}[language=Java]
function gererLumiere(action) {
    const corpsPage = document....................;

    if (action === 'eteindre') {
        corpsPage.classList....................("mode-nuit");
    } else {
        corpsPage.classList.remove("mode-nuit");
    }
}
\end{lstlisting}

%%%%%%%%%%%%%%%%%%


\end{document}