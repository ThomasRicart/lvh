\begin{rubric}{Formations}
\entry*[2019 -- 2020]%
	\textbf{Diplôme Inter Universitaire :} Enseigner l'informatique au lycée. 
%
\entry*[2014]%
	\textbf{CAPES externe de Mathématiques} 

\entry*[2006]%
	\textbf{CAPET externe de Technologie}

\entry*[2004 -- 2006]%
	\textbf{Préparation au concours externe de l'agrégation de Génie Électrique}: Biadmissible 2005 et 2006.\@

\entry*[2001 -- 2004]%
	\textbf{Doctorat ``Conception et réalisation d'éléments de stockages pour convertisseurs statiques intégrés''} Laboratoire d'Analyse et d'Architecture des Systèmes (LAAS -- CNRS -- Toulouse) -- groupe TMN (C. Alonso et C. Vieu).\@

%Utilisation de logiciel de simulation aux éléments finis, Mathlab, travail en salle blanche \ldots
	\entry*[2000 -- 2001]%
	\textbf{DEA ``Conception de Circuits Microélectronique et Microsytèmes CCMM''}: Laboratoire d'Analyse et d'Architecture des Systèmes (LAAS -- CNRS -- Toulouse) dans le groupe TMN (C. Alonso).\@ 

	2001: Stage groupe TMN: Conception et réalisation d'éléments de stockage magnétiques en vue de réaliser des convertisseurs intégrés (Rapport 01309).\@

	\entry*[1999 -- 2000]%
	\textbf{Maitrise ``Électronique, Électrotechnique et Automatisme''}: Université Joseph Fourier (Grenoble -- 38).\@

	2000: Stage au Laboratoire de Technique de l'Information et de la Microélectronique pour l'Architecture des Ordinateurs (TIMA).\@

\entry*[1996 -- 1999]%
	\textbf{Élève ingénieur à l'École Nationale Supérieure d'Électronique et de Radioélectricité (ENSERG):} Institut National Polytechnique (Grenoble -- 38).\@

	1998: Stage au Laboratoire de Physique des Composants à Semi-conducteur (LPCS -- Grenoble).\@

\entry*[1994 -- 1996]%
	\textbf{Classe Préparatoire Polytechnique (CPP) intégrée à l'Institut National Polytechnique (INP -- Toulouse -- 31)}.\@


	\entry*[1994]%
	\textbf{Bac Scientifique avec Mention}: Lycée Bellevue (Toulouse -- 31).\@


	\end{rubric}