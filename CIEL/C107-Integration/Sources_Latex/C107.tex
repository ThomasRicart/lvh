\def\chaptercode{C107}    % Code du chapitre
\def\chaptername{Intégration}        % Nom du chapitre

\titre{}

\medskip

\begin{definition}{Unité d'aire}{}

    \begin{minipage}{0.79\linewidth}
        Soit \Oij{} un repère orthogonal du plan. On note $I$ et $J$ les points tels que $\vec{OI}=\vec{\imath}$ et $\vec{OJ}=\vec{\jmath}$.

        \smallskip

    L'\textbf{unité d'aire}, que l'on note \verb|u.a.| est l'aire du rectangle dont $O$, $I$ et $J$ forment trois sommets.

    \end{minipage}\hfill
    \begin{minipage}{0.19\linewidth}
        \includegraphics[width=1\linewidth]{01}
    \end{minipage}
\end{definition}

\medskip

\section{Intégrale d'une fonction continue et positive}

\begin{definition}{Notion d'intégrale}{}

    \begin{minipage}{0.69\linewidth}
        Soit $f$ une fonction continue et positive sur un intervalle $[a;b]$ de courbe représentative $\mathcal{C}_f$ dans un repère orthogonale \Oij.

        \smallskip

        L'\textbf{intégrale} de $a$ à $b$ de $f$ est l'aire, exprimée en unité d'aires, du domaine situé entre la courbe $\mathcal{C}_f$, l'axe des abscisses et les droites d'équation $x=a$ et $x=b$

        \smallskip

        Cette aire se note $\boxed{\displaystyle\int_{a}^{b} \, f(x)dx}$ et se lit ``intrégrale de $a$ à $b$ de $f(x)$ $dx$''
    \end{minipage}\hfill
    \begin{minipage}{0.29\linewidth}
        \includegraphics[width=1\linewidth]{02}
    \end{minipage}
\end{definition}




\begin{remarque}{}{}
\begin{itemize}
    \item $a$ et $b$ sont les \textbf{bornes inférieures et supérieures} de l'intégrale. Par défaut $a\leq b$.
    \item $x$ est une variable muette. Cette variable n'intervient pas dans le résultat final. On peut remplacer $x$ par $y$, $t$\ldots
    \item Pour tout fonction positive: $\displaystyle\int_{a}^{a}\, f(x)dx = 0$ (Aire d'un segment de hauteur $f(a)$)
    \item Le symbôle $\int$ est dû à G.W. Leibniz (1646--1716). Il ressemble à un ``s'' allongé, rappelant que l'aire peut être calculée comme la somme de petites aires élémentaires.
\end{itemize}

\end{remarque}



\begin{example}{}{}
    
    \begin{minipage}{0.79\linewidth}
        Soit $f$ la fonction définie par $f(x)=\dfrac{x}{2}+2$. Le domaine colorié est un trapèze dont l'aire est:
    
    \vspace{1cm}

    Les unités graphiques sont OI=0,6 cm et OJ=0,8 cm. Quelle est la valeur de l'aire coloriée?

    \bigskip
    
    \end{minipage}\hfill
    \begin{minipage}{0.19\linewidth}
        \includegraphics[width=1\linewidth]{03}
    \end{minipage}

\end{example}


\begin{exercice}{Identification d'une aire à l'aide d'une intégrale}{}

    \begin{minipage}{0.3\linewidth}
       
    Dans chacune des situations, hachurer la partie du plan dont l'intégrale donne l'aire.
    
    \medskip

    Donner un encadrement le plus précis possible pour la valeur de cette intégrale en \verb|u.a.|
    \end{minipage}\hfill
    \begin{minipage}{0.69\linewidth}
        \includegraphics[width=1\linewidth]{10}
    \end{minipage}


\end{exercice}

\pagebreak

\begin{propriete}{Calcul d'une intégrale}{}

    Soit $f$ une fonction continue et positive sur $[a;b]$ et $F$ une primitive de $f$ sur cet intervalle, alors:

    \[
        \boxed{\int_{a}^{b}\, f(x) dx = \left[F(x)\right]{}_{a}^{b}=F(b)-F(a)}
    \]
\end{propriete}

\begin{methode}{Calculer une intégrale}{}

    L'objectif est de calculer une intégrale du type $\int_{a}^{b}\, f(x) dx$
\begin{enumerate}
    \item Identifier les valeurs de $a$ et de $b$
    \item S'assurer que la fonction est positive et continue sur $[a;b]$
    \item Déterminer une primitive $F$ de $f$ sur l'intervalle considéré.
    \item Calculer $F(b)$ et $F(a)$
    \item Calculer l'intégrale par $\int_{a}^{b}\, f(x) dx = \left[F(x)\right]{}_a^b=F(b)-F(a)$
\end{enumerate}
\end{methode}

\begin{exercice}{Calculer les intégrales suivantes}{}

    \begin{multicols}{3}
         \[I_1=\int_{0}^{1} x^2 dx\]

         \[I_2=\int_{1}^{\e} \frac{1}{x}dx\]

         \[I_3=\int_{1}^{3} x(x^2+1){}^2 dx\]

    \end{multicols}


         
\end{exercice}

\medskip

\section{Intégration d'une fonction continue de signe quelconque}

\begin{propriete}{}{}

    Si la fonction $f$ n'est pas positive sur tout l'intervalle $[a;b]$, alors on ne peut PLUS dire que $\int_{a}^{b}f(x)dx$ est l'aire sous la courbe représentative de la fonction $f$.


\end{propriete}


\begin{example}{Calculer $\int_{-1}^{2}\,(x^2-2)dx$ et conclure.}{}

\end{example}

\medskip

\begin{propriete}{}{}

    \begin{itemize}
        \item     Pour tout fonction $f$ continue sur $[a;b]$, on a \[\int_{b}^{a}\, f(x)dx=-\int_{a}^{b}f(x)dx\]
        \item \textbf{Linéarité de l'intégrale}:
        
        \[\int_{a}^{b}\,(f+g)(t)dt = \int_{a}^{b} \, f(t)dt+\int_{a}^{b}g(t)dt \qquad \qquad \qquad \int_{a}^{b}kf(t)dt=k\int_{a}^{b}f(t)dt\]
    \end{itemize}
\end{propriete}

\medskip

\begin{propriete}{Fonction négative et aire}{}

    \begin{minipage}{0.64\linewidth}
    
        Soit $f$ une fonction continue est négative sur un intervalle $[a;b]$. Alors l'aire du domaine situé entre $\mathcal{C}_f$ et l'axe des abscisses sur $[a;b]$ est 

        \medskip

        \[A_\mathcal{D}= A_\mathcal{E}=\int_{a}^{b}(-f(x))dx=-\int_{a}^{b}f(x)dx\]
    \end{minipage}\hfill
    \begin{minipage}{0.34\linewidth}
        \includegraphics[width=1\linewidth]{05}
    \end{minipage}
\end{propriete}

\pagebreak

\begin{exercice}{\quad}{}

    \begin{minipage}{0.69\linewidth}
    
            On considère la fonction $f$ définie par $f(x)=x^2-x-2$ et on note $A$ l'aire du domaine compris entre la courbe $\mathcal{C}_f$, l'axe des abscisses et les droites $x=-1$ et $x=3$.

        \begin{enumerate}
             \item Etudier le signe de $f$ sur $[-1;3]$
           \item Déterminer une primitive $F$ de $f$
           \item Déterminer la valeur de l'aire $A=A_1+A_2$
        \end{enumerate}
        \end{minipage}\hfill
    \begin{minipage}{0.29\linewidth}
        \begin{center}
                    \includegraphics[width=0.8\linewidth]{11}

        \end{center}
    \end{minipage}

\end{exercice}

\begin{propriete}{Relation de Chasles}{}

\begin{minipage}{0.69\linewidth}
    
            Soient $f$ une fonction continue sur un intervalle $I$ et $a$; $b$ et $c$ trois réels appartenant à $I$. Alors:

    \[\int_{a}^{c}f(x)dx=\int_{a}^{b}f(x)dx+\int_{b}^{c}f(x)dx\]
        \end{minipage}\hfill
    \begin{minipage}{0.29\linewidth}
        \begin{center}
                    \includegraphics[width=0.95\linewidth]{06}

        \end{center}
    \end{minipage}


    
\end{propriete}

\medskip

\section{Aire entre deux courbes}

\begin{propriete}{Aire entre deux courbes}{}

    \begin{minipage}{0.69\linewidth}
    
            Soient $f$ et $g$ deux fonctions continues et positives sur un intervalle $[a;b]$ telles que sur cet intervalle, $f(x) \geq g(x)$. Alors l'aire du domaine compris entre les courbes $\mathcal{C}_f$ et $\mathcal{C}_g$ sur l'intervalle $[a;b]$ est donné par \[\boxed{\int_{a}^{b}(f-g)(x)dx}\]
        
        
        \end{minipage}\hfill
    \begin{minipage}{0.29\linewidth}
        \begin{center}
                    \includegraphics[width=0.85\linewidth]{12}

        \end{center}
    \end{minipage}
    
\end{propriete}

\begin{exercice}{\quad}{}

    \begin{center}
        \includegraphics[width = 0.75\linewidth]{a1}

    \end{center}
\end{exercice}

\pagebreak

\begin{exercice}{\quad}{}

\includegraphics[width = 0.95\linewidth]{a3}
\end{exercice}

\medskip

\section{Valeur moyenne}

\begin{definition}{Valeur moyenne}{}


    Soit $f$ une fonction continue sur un intervalle $[a;b]$. La \textbf{valeur moyenne} de $f$ sur $[a;b]$ est le nombre $\mu$ défini par:

    \[\boxed{\mu=\dfrac{1}{b-a}\int_{a}^{b} f(t)dt}\]
\end{definition}


\begin{remarque}{\quad}{}

    \begin{minipage}{0.69\linewidth}
    
Dans le cas où $f$ est positive et continue sur $[a ; b]$, la
valeur moyenne de $f$ entre $a$ et $b$ représente la hauteur
du rectangle construit sur l'intervalle $[a ; b]$.\\
L'aire du rectangle ABCD est égale, en u.a., à l'aire du
domaine coloré car d'après la définition:
\[\mu(b-a)=\int_{a}^{b}f(t)dt\]        
        
        \end{minipage}\hfill
    \begin{minipage}{0.29\linewidth}
        \begin{center}
                    \includegraphics[width=1\linewidth]{09}

        \end{center}
    \end{minipage}

    
\end{remarque}


\begin{exercice}{Calculer la valeur moyenne de la fonction sur l'intervalle considéré}{}
\begin{enumerate}
    \item $f(x)=7x^2-2x+4$ sur $[1;3]$
    \item $f(x)=x^3$ sur $[-3;3]$
    \item $f(x)=2x\e^{x^2}$ sur $[2;3]$
\end{enumerate}

\end{exercice}

\resetalltcb{}
