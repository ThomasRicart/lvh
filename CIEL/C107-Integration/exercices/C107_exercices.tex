\documentclass[a4paper,10pt]{report}
\def\packagepath{../../../../../preambule} 
\usepackage{\packagepath/preambule}

\def\level{BTS }
\def\course{CIEL1}
\def\eval{Intégration}
\def\date{C107 -- TD1}
\begin{document}

\pagestyle{DS_LP}

\NomPrenom{}

\bigskip

%%=============================================================
\exods{0} \medskip

\textbf{Calculs d'intégrales basiques}: 
Calculer les intégrales suivantes en utilisant les primitives usuelles :
\begin{enumerate}
    \item $I_1 = \int_{1}^{2} (3x^2 - 2x) dx$
    \item $I_2 = \int_{0}^{\pi} \cos(t) dt$
    \item $I_3 = \int_{0}^{1} e^{3t} dt$
\end{enumerate}
\dotfill{}

\bigskip

%%=============================================================
\exods{0} \medskip

\textbf{Valeur moyenne et exponentielle composée}: 
Soit la fonction $f$ définie sur $[0 ; 2]$ par : $f(t) = 4t e^{t^2}$.
\begin{enumerate}
    \item Justifier que $F(t) = 2e^{t^2}$ est une primitive de $f$ sur $[0 ; 2]$.
    \item Calculer la valeur exacte de l'intégrale $I = \int_{0}^{2} 4t e^{t^2} dt$.
    \item En déduire la valeur moyenne $\mu$ de la fonction $f$ sur l'intervalle $[0 ; 2]$. 
\end{enumerate}
\dotfill{}

\bigskip

%%=============================================================
\exods{0} \medskip

\textbf{Application au signal (Valeur efficace)}: 
En électronique, la valeur efficace d'un signal $u(t)$ est liée à l'intégrale de son carré. On considère $f(t) = \sin^2(t)$ sur $[0 ; \pi]$.
\begin{enumerate}
    \item À l'aide de la formule $\sin^2(t) = \frac{1-\cos(2t)}{2}$, calculer $\int_{0}^{\pi} \sin^2(t) dt$.
    \item En déduire la valeur moyenne du carré du signal sur une demi-période.
\end{enumerate}
\dotfill{}

\bigskip

%%=============================================================
\exods{0} \medskip

\textbf{Interprétation graphique}: 
Soit la fonction $f(x) = 4-x^2$.
\begin{enumerate}
    \item Calculer l'intégrale $J = \int_{-2}^{2} (4-x^2) dx$.
    \item Sachant que $1 \text{ u.a.} = 2 \text{ cm}^2$, déduire l'aire du domaine sous la courbe en $\text{cm}^2$.
\end{enumerate}
\dotfill{}

\end{document}