\documentclass[a4paper,10pt]{report}

\def\packagepath{../../../../../preambule}   % path du package principal
\usepackage{\packagepath/preambule}    % utilisation du fichier de configuration

\def\level{NSI }              % Classe
\def\course{Première}           % Matière
\def\eval{S38 -- Web -- CSS }

\def\visibleornot{visible}    % visible or invisible
\def\documentpath{./Sources_Latex}
\def\date{}
\renewcommand{\arraystretch}{1}  % Ecart dans les tableaux

\def\ptsexoA{0}
\def\ptsexoB{0}

\def\ptstotal{\ptsexoA+\ptsexoB}

\usepackage{hyperref}

\usetikzlibrary{shapes, arrows, positioning}
\usetikzlibrary{shapes.geometric, arrows, positioning, decorations.pathreplacing}

\tikzstyle{etat} = [draw, rounded corners=15pt, minimum width=2.5cm, minimum height=1.2cm, text centered, font=\bfseries]
\tikzstyle{fleche} = [thick, ->, >=stealth]

\usepackage{float}
\usepackage[T1]{fontenc}

\lstset{
    basicstyle=\ttfamily\small,
    breaklines=true,
    frame=single,
    backgroundcolor=\color{gray!5},
    keywordstyle=\color{blue},
    commentstyle=\color{olive},
    stringstyle=\color{red},
    numberstyle=\tiny\color{darkgray},
    showstringspaces=false,
    tabsize=2
}

\begin{document}

\renewcommand{\labelitemi}{\textbullet}

\pagestyle{DS_LP}
\NomPrenom{}

%%=============================================================
\exods{0}

\medskip


\textbf{Analyse de règles CSS}

Soit la règle CSS suivante appliquée aux sections de votre dashboard :
\begin{lstlisting}
section {
    background-color: white;
    border: 2px solid #2c3e50;
    padding: 15px;
    margin-bottom: 20px;
}
\end{lstlisting}


\begin{enumerate}
    \item \textbf{Identification} : Identifiez le \textbf{sélecteur} utilisé. \dotfill{}
    \item Quelle est la valeur de la couleur de bordure (en hexadécimal) ? \dotfill{}
    \item Quelle propriété permet d'ajouter de l'espace \textbf{à l'intérieur} de la bordure ? \dotfill{}
\end{enumerate}


\bigskip

\exods{0}

\textbf{Le Modèle de Boîte}


Le schéma ci-dessous représente le modèle de boîte d'un bouton.
\begin{center}
    % Simulation visuelle du box model
    \setlength{\unitlength}{1mm}
    \begin{picture}(100,40)
        \put(10,5){\framebox(80,30){}} \put(12,32){Margin}
        \put(20,10){\framebox(60,20){}} \put(22,26){Border}
        \put(30,15){\framebox(40,10){Contenu}} \put(32,21){Padding}
    \end{picture}
\end{center}

Vous voulez que votre bouton ne touche pas le texte voisin et qu'il y ait de l'espace entre son texte "Allumer" et son bord bleu. Quelles propriétés modifiez-vous ?
\begin{itemize}
    \item Pour l'espace extérieur (voisins) : ............................................................
    \item Pour l'espace intérieur (confort) : ............................................................
\end{itemize}

\bigskip

\exods{0}

\textbf{Interactivité et Pseudo-classes}

On souhaite que les boutons du salon changent d'apparence au survol de la souris. 

Compléter le code CSS ci-dessous pour que le fond devienne orange (\verb|#e67e22|) lors du survol:

\begin{lstlisting}
button {
    background-color: #3498db;
    transition: 0.3s; 
}

button:............ {
    background-color: ........................;
}
\end{lstlisting}


\bigskip

\exods{0}

\textbf{Application au Dashboard}

En reprenant la structure de la séance S37 (votre dashboard domotique), écrivez les règles CSS pour remplir ces objectifs :

\begin{enumerate}
    \item Mettre tous les titres \texttt{h1} en bleu foncé et centrés.
    \item Ajouter une bordure arrondie de 10px et une ombre portée aux éléments \texttt{section}.
\end{enumerate}

%%%%%%%%%%%%%%%%%%


\end{document}