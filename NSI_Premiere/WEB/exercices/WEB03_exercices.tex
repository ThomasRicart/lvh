\documentclass[a4paper,10pt]{report}

\def\packagepath{../../../../../preambule}   % path du package principal
\usepackage{\packagepath/preambule}    % utilisation du fichier de configuration

\def\level{NSI }              % Classe
\def\course{Première}           % Matière
\def\eval{WEB03 -- Web -- Interaction client -- serveur }

\def\visibleornot{visible}    % visible or invisible
\def\documentpath{./Sources_Latex}
\def\date{}
\renewcommand{\arraystretch}{1}  % Ecart dans les tableaux

\def\ptsexoA{0}
\def\ptsexoB{0}

\def\ptstotal{\ptsexoA+\ptsexoB}

\usepackage{hyperref}

\usetikzlibrary{shapes, arrows, positioning}
\usetikzlibrary{shapes.geometric, arrows, positioning, decorations.pathreplacing}

\tikzstyle{etat} = [draw, rounded corners=15pt, minimum width=2.5cm, minimum height=1.2cm, text centered, font=\bfseries]
\tikzstyle{fleche} = [thick, ->, >=stealth]

\usepackage{float}
\usepackage[T1]{fontenc}

\lstset{
    language=HTML,
    basicstyle=\ttfamily\small,
    breaklines=true,
    frame=single,
    backgroundcolor=\color{gray!5},
    keywordstyle=\color{blue},
    commentstyle=\color{olive},
    stringstyle=\color{red},
    showstringspaces=false,
    tabsize=2
}


\begin{document}


\renewcommand{\labelitemi}{\textbullet}

\pagestyle{DS_LP}
\NomPrenom{}

%%=============================================================
\exods{0}

\medskip


\textbf{Anatomie d'un formulaire}

Observez l'extrait de code suivant ajouté à votre dashboard :

\begin{lstlisting}
<form action="reglage.html" method="GET">
    <label for="temp">Consigne :</label>
    <input type="range" id="temp" name="v_temp" min="15" max="25">
    <button type="submit">Valider</button>
</form>
\end{lstlisting}


\begin{enumerate}
    \item Quelle est la \textbf{méthode} d'envoi utilisée ? \dotfill{}
    \item Quel est le nom du fichier qui va \textbf{recevoir} les données ? \dotfill{}
    \item À quoi sert l'attribut \texttt{name="v\_temp"} ? \dotfill{}
\end{enumerate}


\bigskip

\exods{0}

\textbf{Analyse d'URL (Méthode GET)}


Un utilisateur utilise le formulaire de l'exercice 1. Il règle le curseur sur \textbf{22} et clique sur le bouton.


\begin{enumerate}
    \item Complétez l'URL telle qu'elle apparaîtra dans la barre d'adresse du navigateur :
    
    \texttt{http://monserveur.fr/reglage.html?............=............}

    \item Si l'utilisateur saisit un \textbf{mot de passe} dans un formulaire en méthode GET, pourquoi est-ce un problème de sécurité ? 
    
\end{enumerate}

\bigskip

\exods{0}

\textbf{GET vs POST}

Complétez le tableau comparatif suivant avec les termes : \textit{URL / Corps de la requête / Limité / Important}.

\begin{center}
\renewcommand{\arraystretch}{1.5}
\begin{tabular}{|l|c|c|}
\hline
\textbf{Caractéristique} & \textbf{Méthode GET} & \textbf{Méthode POST} \\ \hline
Visibilité des données & Dans l'........................ & Dans le ........................ \\ \hline
Volume de données & ........................ & Très important \\ \hline
Sécurité (Mots de passe) & Faible & ........................ \\ \hline
\end{tabular}
\end{center}


\bigskip

\exods{0}

\textbf{Extension du Dashboard}

Vous souhaitez ajouter un champ permettant de choisir la \textbf{couleur de l'éclairage} du salon dans votre section de configuration.


\noindent \textbf{Consigne :} Écrivez le code HTML d'un élément \texttt{<label>} et d'un \texttt{<input>} de type \texttt{color} ayant pour nom (\texttt{name}) "couleur\_led".


%%%%%%%%%%%%%%%%%%


\end{document}