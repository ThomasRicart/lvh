\documentclass[a4paper,10pt]{report}

\def\packagepath{../../../../../preambule}   % path du package principal
\usepackage{\packagepath/preambule}    % utilisation du fichier de configuration

\def\level{NSI }              % Classe
\def\course{Première}           % Matière
\def\eval{WEB04 -- Web -- Java Script 01 }

\def\visibleornot{visible}    % visible or invisible
\def\documentpath{./Sources_Latex}
\def\date{}
\renewcommand{\arraystretch}{1}  % Ecart dans les tableaux

\def\ptsexoA{0}
\def\ptsexoB{0}

\def\ptstotal{\ptsexoA+\ptsexoB}

\usepackage{hyperref}

\usetikzlibrary{shapes, arrows, positioning}
\usetikzlibrary{shapes.geometric, arrows, positioning, decorations.pathreplacing}

\tikzstyle{etat} = [draw, rounded corners=15pt, minimum width=2.5cm, minimum height=1.2cm, text centered, font=\bfseries]
\tikzstyle{fleche} = [thick, ->, >=stealth]

\usepackage{float}
\usepackage[T1]{fontenc}

\lstset{
    language=HTML,
    basicstyle=\ttfamily\small,
    breaklines=true,
    frame=single,
    backgroundcolor=\color{gray!5},
    keywordstyle=\color{blue},
    tagstyle=\color{blue},
    commentstyle=\color{green!50!black},
    stringstyle=\color{red},
    tabsize=2,
    showstringspaces=false
}

\begin{document}

\renewcommand{\labelitemi}{\textbullet}

\pagestyle{DS_LP}
\NomPrenom{}

%%=============================================================
\exods{0}

\medskip


\textbf{Analyse de code et Syntaxe JS}: Observez le script suivant destiné à gérer l'éclairage du salon :

\begin{lstlisting}[language=Java]
const piece = "Salon";
let intensite = 50;

function augmenterLumiere() {
    intensite = intensite + 10;
    console.log("Nouvelle intensite au " + piece + " : " + intensite);
}
\end{lstlisting}

\begin{enumerate}
    \item \textbf{Identification} : Identifiez une \textbf{variable} dont la valeur peut changer et une \textbf{constante} dans ce code.
       
    Variable : .........................................................  Constante : .........................................................
    
    \item \textbf{Type de données} : Quel est le type de la variable \texttt{piece} ? (Nombre ou Chaîne de caractères) \dotfill{}
    
    \item \textbf{Fonction} : Quel est le nom de la fonction définie dans ce script ? \dotfill{}
\end{enumerate}

\bigskip

\exods{0}

\textbf{Manipulation du DOM}: Le \textbf{DOM} (\textit{Document Object Model}) permet à JavaScript de modifier le HTML. 

Reliez chaque instruction à son action correspondante :

\begin{center}
\renewcommand{\arraystretch}{1.5}
\begin{tabular}{|l|c|l|}
\hline
\textbf{Instruction} & & \textbf{Action} \\ \hline
\texttt{document.getElementById("id")} & $\bullet$ \hspace{1cm} $\bullet$ & \textbf{A.} Modifie la couleur du texte. \\ \hline
\texttt{.textContent = "..."} & $\bullet$ \hspace{1cm} $\bullet$ & \textbf{B.} Cible un élément précis. \\ \hline
\texttt{.style.color = "..."} & $\bullet$ \hspace{1cm} $\bullet$ & \textbf{C.} Change le texte affiché. \\ \hline
\end{tabular}
\end{center}

\bigskip

\exods{0}

\textbf{Événements et Interactivité}

On souhaite qu'un message de bienvenue s'affiche lorsqu'on clique sur un bouton.

\begin{enumerate}
    \item \textbf{Côté HTML} : Complétez la balise pour appeler la fonction \texttt{direBonjour()}.
    \begin{lstlisting}
<button ....................="direBonjour()">Connexion</button>
    \end{lstlisting}

    \item \textbf{Côté JavaScript} : Complétez la fonction pour que le texte de l'élément d'ID \texttt{message} devienne "Systeme Pret".
    \begin{lstlisting}[language=Java]
function direBonjour() {
    const affichage = document.getElementById("....................");
    affichage.................... = "Systeme Pret";
}
    \end{lstlisting}
\end{enumerate}

\bigskip

\exods{0}

\textbf{Application au Dashboard (Mini-projet)}

Vous voulez créer un bouton "Mode Nuit" qui, lors d'un clic, passe le fond du Dashboard en noir et le texte en blanc. 

\textbf{Consigne :} Écrivez le code JavaScript nécessaire pour modifier le style de l'élément ayant l'ID \texttt{"dashboard-bg"}.

\begin{lstlisting}[language=Java]
function activerModeNuit() {
    // 1. Recuperer l'element d'ID "dashboard-bg"
    const ecran = ............................................................;

    // 2. Changer la couleur de fond (backgroundColor) en "black"
    ecran.style......................... = "black";

    // 3. Changer la couleur du texte (color) en "white"
    ..........................................................................;
}
\end{lstlisting}

%%%%%%%%%%%%%%%%%%


\end{document}