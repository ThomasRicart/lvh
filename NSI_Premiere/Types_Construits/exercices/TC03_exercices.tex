\documentclass[a4paper,10pt]{report}

\def\packagepath{../../../../../preambule}   % path du package principal
\usepackage{\packagepath/preambule}    % utilisation du fichier de configuration

\def\level{NSI }              % Classe
\def\course{Première}           % Matière
\def\eval{TC03 -- Les Dictionnaires }

\def\visibleornot{visible}    % visible or invisible
\def\documentpath{./Sources_Latex}
\def\date{}
\renewcommand{\arraystretch}{1.2}

\def\ptsexoA{5}
\def\ptsexoB{5}
\def\ptsexoC{5}
\def\ptsexoD{5}

\def\ptstotal{\ptsexoA+\ptsexoB+\ptsexoC+\ptsexoD}

\usepackage{hyperref}
\usepackage{float}
\usepackage[T1]{fontenc}

\lstset{
    language=Python,
    basicstyle=\ttfamily\small,
    breaklines=true,
    frame=single,
    backgroundcolor=\color{gray!5},
    keywordstyle=\color{blue},
    commentstyle=\color{green!50!black},
    stringstyle=\color{red},
    tabsize=4,
    showstringspaces=false
}

\begin{document}

\pagestyle{DS_LP}
\NomPrenom{}

%%=============================================================
\exods{0} \medskip

\textbf{Compréhension et Vocabulaire}: Un développeur utilise un dictionnaire pour stocker l'inventaire d'un joueur :
\begin{lstlisting}
inventaire = {"potions": 5, "pieces": 150, "epee": 1}
\end{lstlisting}

\begin{enumerate}
    \item \textbf{Type et Terminologie} : Quel est le type de la variable \texttt{inventaire} ? Dans le couple \texttt{"pieces": 150}, identifiez la clé et la valeur. \dotfill{}
    \item \textbf{Accès} : Quelle instruction permet d'afficher le nombre de \texttt{"potions"} ? \dotfill{}
    \item \textbf{Mise à jour} : Le joueur achète une potion supplémentaire. Écrivez l'instruction qui met à jour la quantité de potions à \texttt{6}. \dotfill{}
    \item \textbf{Ajout} : Le joueur ramasse un nouvel objet : \texttt{"bouclier"} (quantité 1). Écrivez l'instruction permettant d'ajouter cet élément au dictionnaire. \dotfill{}
\end{enumerate}

%%=============================================================
\exods{0} \medskip

\textbf{Parcours et Filtrage}: On considère un dictionnaire de produits et leurs prix en euros :
\begin{lstlisting}
catalogue = {"Clavier": 45, "Souris": 25, "Ecran": 180, "Tapis": 15}
\end{lstlisting}

Complétez le script Python suivant pour qu'il affiche uniquement les \textbf{noms} des produits dont le prix est \textbf{inférieur à 30 euros}.

\begin{lstlisting}
for produit, prix in catalogue..........:
    if .......... < 30:
        print(..........)
\end{lstlisting}

%%=============================================================
\exods{0} \medskip

\textbf{Algorithme de Fréquences} 

Cet algorithme classique permet de compter les occurrences de chaque caractère dans une chaîne. Complétez les zones vides.

\begin{lstlisting}
texte = "nsi"
frequences = {}

for caractere in texte:
    if caractere in ..........:
        frequences[caractere] = frequences[caractere] + 1
    else:
        frequences[caractere] = ..........

print(frequences) # Doit afficher {'n': 1, 's': 1, 'i': 1}
\end{lstlisting}

%%=============================================================
\exods{0} \medskip

\textbf{Structures Complexes : Bibliothèque}: On utilise une liste de dictionnaires pour gérer des livres :
\begin{lstlisting}
biblio = [
    {"titre": "1984", "auteur": "Orwell", "disponible": True},
    {"titre": "Dune", "auteur": "Herbert", "disponible": False},
    {"titre": "Fondation", "auteur": "Asimov", "disponible": True}
]
\end{lstlisting}

\begin{enumerate}
    \item Quelle instruction permet d'accéder au dictionnaire du livre "Dune" ?
    \item Que renvoie l'instruction \texttt{biblio[2]["auteur"]} ?
    \item \textbf{Écriture de fonction} : Complétez la fonction ci-dessous qui renvoie \texttt{True} si un livre est disponible, et \texttt{False} sinon.
\end{enumerate}

\begin{lstlisting}
def est_dispo(titre_recherche, table_livres):
    for livre in table_livres:
        if livre["titre"] == ..........:
            return ..........["disponible"]
    return "Livre non reference"
\end{lstlisting}

\end{document}