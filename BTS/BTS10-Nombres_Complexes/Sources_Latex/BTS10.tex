\def\chaptercode{BTS10}    % Code du chapitre
\def\chaptername{Nombres complexes}

\titre{}

\medskip
\begin{propriete}{On considère un point $M(x;y)$ et $\alpha \in \left]0;\frac{\pi}{2}\right[$}{}


\begin{minipage}{0.2\linewidth}
	\begin{itemize}
		\item $\cos \alpha = x = \dfrac{OH}{OM}$
		\item $\sin \alpha = x = \dfrac{HM}{OM}$
	\end{itemize}
\end{minipage}\hfill
\begin{minipage}{0.20\linewidth}
	\includegraphics[width = 0.8\linewidth]{04}
\end{minipage}
\begin{minipage}{0.59\linewidth}
\includegraphics[width = 0.95\linewidth]{03}
\end{minipage}
\end{propriete}

\section{Approche algébrique : L'unité $\i$}

\begin{definition}{Nombre complexe\textendash{}forme algébrique}{}\label{def:forme_algebrique}
	Il existe un ensemble $\C$ tel que :
	\begin{enumerate}[1)]
		\item $\C$ contient $\R$ et un élément $\i$ tel que \fbox{$\i^2=-1$}.
		\item Tout nombre complexe s'écrit \fbox{$z=a+\i b$} ($a, b \in \R$).
	\end{enumerate}
\end{definition}

\begin{exercice}{Calculs immédiats}{}
	Calculer et simplifier pour mettre sous la forme $a + \i b$ :
	\begin{itemize}
		\item $A = (3+5\i) - (2-4\i) = \dots$
		\item $B = \i(3-4\i)$
		\item $C = (1+\i)^2 = \dots$
	\end{itemize}
\end{exercice}

\begin{exercice}{Calcul de $i^n$}
	\begin{itemize}
		\item Calculer les valeurs de $i^2$, $i^3$, $i^4$, $i^5$ ...
		\item Sur Géogébra
		\begin{itemize}
			\item placer un curseur $n$ allant de $-10$ à $10$ par pas de $1$
			\item dans la partie calcul formel, taper $\i^n$. (Attention le $\i$ complexe se forme en appuyant sur \verb|ALT| + \verb|i|)
			\item faire varier $\i$ et observer sa valeur
			\item observer ce qu'il se passe dans la partie graphique
		\end{itemize} 
	\end{itemize}
\end{exercice}

\begin{exercice}{Calculs complexes avec Géogébra}
	Par défaut Géogébra cherche à convertir les nombres complexes sous forme algébrique. Tester le calcul formel et créant les nombres complexes suivants et observez la partie graphique.

	$z_1 = (3+4\i) \times (4-\i)$ \hfill 
	$z_2=\frac{1-3\i}{4-\i}$ \hfill
	$z_3=(1-\i)^6$ \hfill
	\hspace{3cm}
\end{exercice}

\begin{propriete}{Le nombre conjugué}{}
	Le conjugué de $z=a+\i b$ est \fbox{$\conj{z} = a - \i b$}.
	\begin{itemize}
		\item \textbf{Méthode :} Pour calculer un quotient $\dfrac{z_1}{z_2}$, on multiplie le haut et le bas par $\conj{z_2}$.
	\end{itemize}
\end{propriete}

\begin{exercice}{Application}{}
	\begin{itemize}
		\item Déterminer le nombre conjugué de $z=4-3\i$
		\item Ecrire sous forme algébrique le nombre $\dfrac{3-\i}{4-3\i}$
		\item Vérifier avec Géogébra
	\end{itemize}
\end{exercice}

\section{Approche géométrique}

\begin{propriete}{Nombres complexes et géométrie}{}

	\begin{minipage}{0.6\linewidth}
 Au point $M(a;b)$, on peut associer le nombre complexe $z=a+\i b$.

 \smallskip
	
	On dit que $z=a+\i b$ est l'\textbf{affixe} du point $M$. On note \[\boxed{M(a+ib)}\]

	Soient $A$ et $B$ les points d'affixes respectifs $z_A$ et $z_B$. 

\begin{itemize}
	\item Le vecteur $\overrightarrow{AB}$ a pour affixe  \qquad \fbox{$z_{AB}=z_B-z_A$}
	\item Le milieu $I$ de $[AB]$ a pour affixe \qquad \fbox{$z_I=\dfrac{z_A+z_B}{2}$} \qquad 
\end{itemize}

\end{minipage}\hfill
\begin{minipage}{0.39\linewidth}
\begin{center}
\includegraphics[width = 0.9\linewidth]{cconj.jpg}
\end{center}
\end{minipage}

\end{propriete}


\begin{exercice}{Nombres complexes et géométrie}{}
\begin{itemize}
	\item Sur Géogébra, placez le point $A$ d'affixe $z_A=1+2\i$
	\item Placez le point $B$ d'affixe $\conj{z_A}$ et le point $C$ d'affixe $-z_A$.
	\item Quelle symétrie permet de passer de A à B? De A à C?
\end{itemize}

\end{exercice}
\section{Résolutions d'équations}

\begin{propriete}{Résolution d'équations du premier degré dans $\C$}{}

	Les résolutions respectent exactement le même protocole que dans $\R$:

	\begin{enumerate}[1)]
		\item Faire passer dans le membre de gauche tout ce qui dépend de $z$ et dans le membre de droite tout ce qui ne dépend pas de $z$.
		\item Simplifier le membre de gauche et de droite
		\item Isoler $z$ et répondre à la question
	\end{enumerate}
\end{propriete}


\begin{exercice}{Résoudre dans $\C$ l'équation $(E): \,3z-6=4\i+z+2$}{}

\end{exercice}

\begin{propriete}{Résolution d'équations du deuxième degré dans $\C$}{}

	Les résolutions respectent exactement le même protocole que dans $\R$:

	\begin{enumerate}[1)]
		\item Mettre l'équation sous la forme $az^2+bz+c=0$ avec $(a, b, c)$ trois réels.
		\item Calculer le \textbf{discriminant} \fbox{$\Delta = b^2-4ac$}
		\begin{itemize}
			\item Si le discriminant est positif alors il existe deux solutions réelles (éventuellement égales).
			
				\[\boxed{z_1=\dfrac{-b-\sqrt{\Delta}}{2a} \quad \textrm{et} \quad z_2=\dfrac{-b+\sqrt{\Delta}}{2a}} \]

			\item Si le discriminant est strictement négatif alors il existe deux solutions complexes conjuguées:
			
	\[\boxed{z_1=\dfrac{-b-\i\sqrt{-\Delta}}{2a} \quad \textrm{et} \quad z_2=\conj{z_1}=\dfrac{-b+\i\sqrt{-\Delta}}{2a}} \]
		\end{itemize}

	\end{enumerate}
\end{propriete}



\section{Formes trigonométrique et exponentielle}

\begin{definition}{Module et argument d'un nombre complexe}{}
	
	\begin{minipage}{0.8\linewidth}
Le \textbf{module} d'un nombre complexe $z=a+\i b$ est le réel positif  \[\boxed{|z|=\sqrt{a^2+b^2}}\]

\smallskip
Un \textbf{argument} de $z$, noté $\arg(z)$ est une mesure exprimée en radian de l'angle \[\theta=\arg(z)=(\vec{u},\vec{OM})\]

\end{minipage}
\begin{minipage}{0.19\linewidth}
\begin{flushright}
		\includegraphics[width = 0.9\linewidth]{02}
\end{flushright}
\end{minipage}

\end{definition}


\begin{definition}{Forme trigonométrique d'un nombre complexe}{}

	Tout nombre complexe $z$ s'écrit sous la \textbf{forme algébrique} $z=a+\i b$. On note $\theta = \arg(z)$

	\medskip


		\[|z|=\sqrt{a^2+b^2} \qquad \qquad \cos \theta = \dfrac{a}{|z|} \qquad \qquad \sin \theta = \dfrac{b}{|z|}\]

		Tout nombre complexe peut donc s'écrire sous la \textbf{forme trigonométrique}: \[\boxed{z=|z|\left(\cos \theta + \i \sin \theta \right)}\]
\end{definition}

\begin{methode}{Passer de la forme algébrique à la forme trigonométrique}{}

	On considère un nombre complexe $z$ écrit sous forme algébrique $z=a+\i b$ 

	\begin{enumerate}[1)]
		\item Calculer $|z|=\sqrt{a^2+b^2}$
		\item Calculer $\cos \theta$ et $\sin \theta$ et trouver la valeur de $\theta$ respectant les deux conditions précédentes.
		\item Ecrire $z=|z|\left(\cos \theta + \i \sin \theta \right)$
	\end{enumerate}

\end{methode}

\begin{exercice}{Ecrire les nombres complexes suivants sous forme trigonométriques}{}

	\begin{enumerate}[1)]

		\begin{multicols}{3}
		\item $z=\frac{\sqrt{3}}{2}-\i \times \frac{1}{2}$
		\item $z= -2+2 \i$
		\item $z= 1+\i\sqrt{3}$
		\end{multicols}
	\end{enumerate}

\end{exercice}

\begin{methode}{Passer de la forme trigonométrique à la forme algébrique}{}

	On considère un nombre complexe $z$ tel que l'on connait $|z|$ et un argument $\theta=\arg(z)$. 

	\begin{enumerate}[1)]
		\item Calculer $a=|z|\cos\theta$
		\item Calculer $b=|z|\sin\theta$
		\item Ecrire $z=|z|\cos\theta + \i |z|\sin\theta$
	\end{enumerate}

\end{methode}


\begin{exercice}{Ecrire les nombres complexes suivants sous forme arithmétique}{}

	\begin{enumerate}[1)]

		\begin{multicols}{3}
		\item $|z|=2$ et $\arg(z)=-\frac{\pi}{3}$
		\item $|z|=\frac{1}{2}$ et $\arg(z)=\frac{5\pi}{6}$
		\item $|z|=3$ et $\arg(z)=\frac{2\pi}{3}$
		\end{multicols}
	\end{enumerate}

\end{exercice}

%%%%%%%%%%%%%

\begin{propriete}{Notation exponentielle (Euler)}{}
	On écrit : \fbox{$z = r\e^{\i\theta}$} avec $r = |z|$ et $\theta = \arg(z)$.
\end{propriete}


\begin{propriete}{Propriétés de la forme exponentielle}{}

\begin{enumerate}[1)]
	\begin{multicols}{3}
	\item $\e^{\i \times 0}=1$
	\item $\e^{\i  \frac{\pi}{2}}=0$
 	\item $\e^{\i \pi}=1$

	\end{multicols}

	\begin{multicols}{4}
	\item $\e^{\i \theta}\times \e^{\i \theta'}=\e^{\i (\theta+\theta')}$
	\item $\frac{1}{\e^{\i \theta}}=\e^{-\i \theta}$
	\item $\frac{\e^{\i \theta}}{\e^{\i \theta'}}=\e^{\i (\theta-\theta')}$
	\item $\left(\e^{\i \theta}\right){}^n=\e^{\i n\theta}$

	\end{multicols}
\end{enumerate}

\end{propriete}

\begin{exercice}{Ecriture sous forme exponentielle}{}

	Ecrire les nombres de l'exercice 7 sous forme exponentielle

\end{exercice}
\section{Application aux circuits}

\begin{exercice}{Impédances complexes}{}
	En électronique, on utilise les nombres complexes pour modéliser les composants :
    \begin{itemize}
        \item Résistance : $Z_R = R$
        \item Bobine : $Z_L = \i L\omega$
    \end{itemize}
    Calculer l'impédance totale $Z = Z_R + Z_L$ sous forme exponentielle pour $R=10\Omega$ et $L\omega=10\Omega$.
\end{exercice}

\pagebreak

\section{Exercices}



\begin{multicols}{2}

	\includegraphics[width = 0.8\linewidth]{a1}

	\includegraphics[width = 0.8\linewidth]{a2}

	\includegraphics[width = 0.8\linewidth]{a3}

	\includegraphics[width = 0.8\linewidth]{a5}

	\includegraphics[width = 0.8\linewidth]{a6}

	\includegraphics[width = 0.8\linewidth]{a8}

	\includegraphics[width = 0.8\linewidth]{a20}

	\includegraphics[width = 0.8\linewidth]{a10}

	\includegraphics[width = 0.8\linewidth]{a11}






	\includegraphics[width = 0.8\linewidth]{a17}

\end{multicols}

\end{document}