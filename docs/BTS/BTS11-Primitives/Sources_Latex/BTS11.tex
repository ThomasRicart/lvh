\def\chaptercode{BTS11}    % Code du chapitre
\def\chaptername{Primitives}        % Nom du chapitre

\titre{}

\medskip

\begin{definition}{Primitive d'une fonction}{}

    Soit $f$ une fonction définie et continue sur un intervalle $I$. On appelle \textbf{primitive} de $f$ sur $I$, toute fonction $F$ définie et dérivable sur $I$ telle que $\boxed{F'=f}$
    
\end{definition}


\begin{example}{Recherche de primitive de $f(x)=2x+1$}{}


    \begin{itemize}
        \item La fonction $F$ définie sur $\R$ par $F(x)= \ldots\ldots\ldots\ldots$ est une primitive de $f$ car $F'(x)=f(x)$.
         \item La fonction $G$ définie sur $\R$ par $G(x)= \ldots\ldots\ldots\ldots$ est une primitive de $f$ car $G'(x)=f(x)$.
   \end{itemize}
\end{example}


\begin{methode}{Montrer qu'une fonction $F$ est une primitive de $f$}{}

\begin{enumerate}
    \item Dériver la fonction $F$.
    \item Montrer que $F'(x)=f(x)$
    \item Conclure
\end{enumerate}

\end{methode}


\begin{exercice}{Exercice d'application}{}

    \begin{enumerate}
        \item Montrer que la fonction $F$ définie par $F(x)=\frac{1}{3}x^3+5$ est une primitive de $f(x)=x^2$.
        \item Montrer que la fonction $G$ définie par $G(x)=\frac{-3}{x^2}+2$ est une primitive de $g(x)=\frac{6}{x^3}$.
        \item Montrer que la fonction $H$ définie par $H(x)=2\e^{-6x}$ est une primitive de $h(x)=-12\e^{-6x}$.
    \end{enumerate}

\end{exercice}


\begin{propriete}{Primitives usuelles}{}

\renewcommand{\arraystretch}{2.3} % Espacement vertical important pour le confort visuel

% --- Premier tableau (Usuelles) ---
\begin{minipage}[t]{0.52\textwidth}
\centering
\textbf{1. Fonctions Usuelles}
\smallskip

\begin{tabular}{|m{2.5cm}|m{3cm}|m{3.5cm}|}
\hline
\textbf{Fonction $f(x)$} & \textbf{Primitive $F(x)$} & \textbf{Validité} \\
\hline
$a$ & $ax + k$ & $\mathbb{R}$ \\
\hline
$x^n$ ($n \in \mathbb{N}$) & $\dfrac{x^{n+1}}{n+1} + k$ & $\mathbb{R}$ \\
\hline
$\dfrac{1}{x^2}$ & $-\dfrac{1}{x} + k$ & $\mathbb{R}^*$ \\
\hline
$\dfrac{1}{x}$ & $\ln(|x|) + k$ & $\mathbb{R}^*$ \\
\hline
$\dfrac{1}{\sqrt{x}}$ & $2\sqrt{x} + k$ & $]0 ; +\infty[$ \\
\hline
$e^x$ & $e^x + k$ & $\mathbb{R}$ \\
\hline
$\cos(x)$ & $\sin(x) + k$ & $\mathbb{R}$ \\
\hline
$\sin(x)$ & $-\cos(x) + k$ & $\mathbb{R}$ \\
\hline
\end{tabular}
\end{minipage}
%
\hfill % Espace entre les deux minipages
%
% --- Deuxième tableau (Composées) ---
\begin{minipage}[t]{0.45\textwidth}
\centering
\textbf{2. Formes Composées}
\smallskip

\begin{tabular}{|m{2.5cm}|m{3.5cm}|}
\hline
\textbf{Forme de $f$} & \textbf{Primitive $F$} \\
\hline
$u'u^n$ ($n \in \mathbb{N}$) & $\dfrac{u^{n+1}}{n+1} + k$ \\
\hline
$\dfrac{u'}{u^2}$ & $-\dfrac{1}{u} + k$ \\
\hline
$\dfrac{u'}{u^n}$ ($n \geq 2$) & $-\dfrac{1}{(n-1)u^{n-1}} + k$ \\
\hline
$\dfrac{u'}{\sqrt{u}}$ & $2\sqrt{u} + k$ \\
\hline
$\dfrac{u'}{u}$ & $\ln(|u|) + k$ \\
\hline
$u'e^u$ & $e^u + k$ \\
\hline
$u' \cos(u)$ & $\sin(u) + k$ \\
\hline
$u' \sin(u)$ & $-\cos(u) + k$ \\
\hline
\end{tabular}
\end{minipage}
\end{propriete}


   

\begin{propriete}{Non unicité des primitives}{}

Soit $f$ une foncton définie et continue sur un intervalle $I$

\begin{itemize}
    \item Si $F$ est une primtive de $f$ alors $f$ admet une \textbf{infinité} de primitives.
    \item Toute fonction $G$ définie par $G(x)=F(x)+k$ est également une primitive de $f$. ($k \in\R$)
\end{itemize}

\end{propriete}


\begin{methode}{Recherche de toutes les primitives de $f$}{}

    \begin{enumerate}
        \item trouver UNE primitive de $f$ et la noter $F$
        \item ajouter une constante $k\in\R$ pour trouver TOUTES les primitives de $f$ et la noter $G$.
    \end{enumerate}
\end{methode}

\medskip

\begin{exercice}{Déterminer toutes les primitives des fonctions suivanes}{}

    \begin{enumerate}
        \item  $f(x)=2x^3+3$
        \item $f(x)=x^2+4x+1$
        \item $f(x)=4\e^{2x}$
    \end{enumerate}
\end{exercice}




\medskip

\begin{propriete}{Unicité de la primitive}{}

    Soit $f$ une fonction continue sur un intervalle $I$ et soient $a$ et $b$ deux réels.

    \smallskip{}

    Il existe une \textbf{unique} primitive $H$ de $f$ vérifiant $H(a)=b$


\end{propriete}

\medskip

\begin{methode}{Déterminer LA primitive vérifiant une condition}{}

    \begin{enumerate}
    \item Déterminer UNE primitive de la fonction $f$ et la noter $F$. $F'(x)=f(x)$
    \item Ajouter une constante $k$ pour trouver TOUTES les primitive. $G(x)=F(x)+k$
    \item Déterminer la valeur de la constante de manière à ce que $H(a)=F(a)+k=b$
    \item Conclure en donnant l'expression de $H(x)$.
\end{enumerate}
\end{methode}

\medskip

\begin{exercice}{Exercice d'application}{}
    Soit $f$ la fonction définie sur $\R$ par $f(x)=x^2+3x+1+\e^{-x}$

    Déterminer la primitive $H$ de $f$ qui est telle que $H(0)=4$.

\end{exercice}

\medskip

\begin{exercice}{Déterminer la primitive $H$ des fonctions suivantes qui respecte la condition donnée}{}

    \begin{itemize}
        \item $f(x)=x^3-1$ et $H(3)=0$
        \item $f(x)=\frac{2}{x}$ et $H(6)=1$
        \item $f(x)=2x(x^2+1){}^2$ et $H(0)=4$
        \item $f(x)=\e^{1-2x}$ et $H(0)=2$
    \end{itemize}

\end{exercice}

\resetalltcb{}
