\documentclass[a4paper,10pt]{report}

\def\packagepath{../../../../../preambule}   % path du package princial
\usepackage{\packagepath/preambule}    % utilisation du fichier de configuration

\def\level{BTS }              % Classe
\def\course{CIEL 1}              % Matière
\def\eval{BTS11 - Primitives - TP1}

\def\visibleornot{visible}    % visible or invisible
\def\documentpath{./Sources_Latex}
\def\date{ }
\renewcommand{\arraystretch}{1}  % Ecart dans les tableau

\def\ptsexoA{0}
\def\ptsexoB{0}
\def\ptsexoC{0}
\def\ptsexoD{4}
\def\ptsexoE{1}
\def\ptsexoF{1}
\def\ptsexoG{2}
\def\ptsexoH{1}
\def\ptsexoI{2}
\def\ptsexoJ{2}
\def\ptsexoK{1}

\def\ptstotal{\ptsexoA+\ptsexoB}

\usepackage{hyperref}

\usetikzlibrary{shapes, arrows, positioning}
\usetikzlibrary{shapes.geometric, arrows, positioning, decorations.pathreplacing}

\tikzstyle{etat} = [draw, rounded corners=15pt, minimum width=2.5cm, minimum height=1.2cm, text centered, font=\bfseries]
\tikzstyle{nouveau} = [etat, fill=blue!40]
\tikzstyle{pret} = [etat, fill=green!60]
\tikzstyle{elu} = [etat, fill=yellow!60]
\tikzstyle{bloque} = [etat, fill=red!60]
\tikzstyle{termine} = [etat, fill=white]
\tikzstyle{fleche} = [thick, ->, >=stealth]

\usepackage{float} % Permet d'utiliser [H]
\usepackage[T1]{fontenc}
\begin{document}

%%%%%%%%%%%%%%%%%%%%%%%%%%%%%%%%%%%%%%%%%%%%%%%%%%ù

%\PageGardeSujetBac[Matiere = Numérique et Sciences Informatiques,
%                  Session = 2025,
%                  AffJour=false,
%                  Duree = {3 heures 30},
%                  DernierePage = \pageref{LastPage},
%                  ModeExamen=false
%                  ]
\renewcommand{\labelitemi}{\textbullet} %pour éviter les tirets dans les "itemize" qui apportent confusion avec le signe moins.
%% Début page de garde style BAC

   %%%%%%%%%%%%%%%%%%%%%%%%%%%%%%%%%%%%%%%%%%%%%%%%%%%%%%%%
%\PageGardeSujetBac[clés]

\pagestyle{DS_LP}          % Feuille de style fancy
\NomPrenomNote{}


\exods{\ptsexoA}

\medskip


\begin{enumerate}
    \item Soit la fonction $f_1$ définie par $f_1(x)=3x^2+2x-1$. 
    \begin{enumerate}
        \item Déterminer une primitive de la fonction $f_1$.
        \item Déterminer la primitive de la fonction $f_1$ qui vaut 6 en $x=0$.
    \end{enumerate}
    \item $f_2(x)=x^3 - 2x^2 + 5x$
    \begin{enumerate}
        \item Déterminer une primitive de la fonction $f_2$.
        \item Déterminer la primitive de la fonction $f_2$ qui vaut 0 en $x=1$.
    \end{enumerate} 
\end{enumerate}

\bigskip

\exods{\ptsexoB}

\medskip

Soit $f$ la fonction définie sur $\R$ par $f(x)=(2x+1)\e^{-x}$.

\begin{enumerate}
    \item Etudier le signe de la fonction $f$ sur $\R$
    \item Etudier les variations de la fonction $f$ sur $\R$.
    \item Déterminer l'équation de la tangente à la courbe représentative de $f$ au point d'abscisse 0.
    \item Montrer que $F(x)=-(2x+3)\e^{-x}$ est une primitive de la fonction $f$.
    \item Déterminer la primitive de la fonction $f$ telle que $F(0)=3$.
\end{enumerate}

\bigskip

\exods{\ptsexoC}

\medskip

\begin{easybox*}{\quad}{}
A faire uniquement si vous 
\begin{itemize}
    \item avez fini les exercices précédents
    \item avez compris parfaitement les exercices précédents
\end{itemize}
\end{easybox*}

\begin{propriete}{Intégration par parties}{}
$$\int u(x)v'(x) dx = u(x)v(x) - \int u'(x)v(x) dx$$

avec
\begin{itemize}
    \item $u$ et $v$ des fonctions dérivables sur un intervalle $I$.
    \item le symbole $\int$ représentera pour le moment le fait de chercher une primitive d'une fonction. 
    \item Ainsi par exemple $\int f(x)dx$ représente une primitive de la fonction $f$.
\end{itemize}



\end{propriete}

\begin{enumerate}
    \item Déterminer une primitive de la fonction $f$ définie par $f(x)=x\e^x$.
    \item Déterminer une primitive de la fonction $f$ définie par $f(x)=x\e^{-x}$.
    \item Déterminer une primitive de la fonction $f$ définie par $f(x)=x\ln(x)$.
    \item Déterminer une primitive de la fonction $f$ définie par $f(x)=\ln(x)$.

\end{enumerate}



\end{document}








